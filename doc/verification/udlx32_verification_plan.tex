\documentclass{article}

\usepackage{ipprocess}
%\usepackage{lmodern}
\usepackage{longtable}
\usepackage[utf8]{inputenc} 
\usepackage[T1]{fontenc}
\pagestyle{fancy}
\usepackage{libertine}
\usepackage{epstopdf}
\usepackage{rotating} % rotate table 90 degrees
\usepackage{pdflscape} % set ladscape/portrait pdf pages

%\usepackage[author={João Carlos Nunes Bittencourt}]{pdfcomment}

\sloppy

\title{32-bit $\mu$DLX Core Processor}

\graphicspath{{./pictures/}} % DiretTBDrio padrTBDo de figuras.
\makeindex
\begin{document}
  \capa{1.0}{abril}{2014}{32-bit $\mu$DLX Core Processor}{Architecture Specification}{Universidade Federal da Bahia}
  \newpage

%%%%%%%%%%%%%%%%%%%%%%%%%%%%%%%%%%%%%%%%%%%%%%%%%%
%% GNU LGPL Licence
%%%%%%%%%%%%%%%%%%%%%%%%%%%%%%%%%%%%%%%%%%%%%%%%%%

  
\begin{center}
\begin{Large}\textbf{GNU LGPL License}\end{Large}
\end{center}
\vspace{2cm}
\fbox{
  \parbox{.7\textwidth}{
    \vspace{0,5cm}
    \begin{scriptsize}
    This file is part of uDLX (micro-DeLuX) soft IP-core.\\

    uDLX is free soft IP-core: you can redistribute it and/or modify
    it under the terms of the GNU General Public License as published by
    the Free Software Foundation, either version 3 of the License, or
    (at your option) any later version.\\

    uDLX soft core is distributed in the hope that it will be useful,
    but WITHOUT ANY WARRANTY; without even the implied warranty of
    MERCHANTABILITY or FITNESS FOR A PARTICULAR PURPOSE. See the
    GNU General Public License for more details.

    You should have received a copy of the GNU General Public License
    along with uDLX. If not, see <http://www.gnu.org/licenses/>.
    \end{scriptsize}
    \vspace{0,5cm}
  }
}

\newpage

%%%%%%%%%%%%%%%%%%%%%%%%%%%%%%%%%%%%%%%%%%%%%%%%%%
%% Revision History
%%%%%%%%%%%%%%%%%%%%%%%%%%%%%%%%%%%%%%%%%%%%%%%%%%
  \section*{\center Revision History}
  \vspace*{1cm}
  \begin{center} % aqui comeTBDa o ambiente tabela
    \begin{longtable}[pos]{|m{2cm} | m{7.2cm} | m{3.8cm}|} 
      \hline % este comando coloca uma linha na tabela
      \cellcolor[gray]{0.9}
      \textbf{Date} & \cellcolor[gray]{0.9}\textbf{Description} & \cellcolor[gray]{0.9}\textbf{Author(s)}\\ \hline
      \hline
      \small 04/27/2014 & \small Conception & \small João Carlos Bittencourt \\ \hline
      \small 04/30/2014 & \small Instruction layout description & \small João Carlos Bittencourt \\ \hline
      \small 05/09/2014 & 
      \begin{small}
        \begin{itemize}
          \item Text revision;
          \item Update diagrams and instruction layout;
          \item Update instruction fetch I/O definitions;
          \item Missing pictures inclusion;
          \item Include memory access and write back pin/port definitions;
        \end{itemize}  
    \end{longtable}
  \end{center}

  \newpage
	%%%%%%%%%%%%%%%%%%%%%%%%%%%%%%%%%%%%%%%%%%%%%%%%%%	
	%% Place TOC
	%%%%%%%%%%%%%%%%%%%%%%%%%%%%%%%%%%%%%%%%%%%%%%%%%%  
  \tableofcontents
  \newpage

	%%%%%%%%%%%%%%%%%%%%%%%%%%%%%%%%%%%%%%%%%%%%%%%%%%
	%% Document Prelimiary Content
	%%%%%%%%%%%%%%%%%%%%%%%%%%%%%%%%%%%%%%%%%%%%%%%%%%
  \section{Introduction}

	\subsection{Purpose}
	
	\subsection{Stakeholders}
  \FloatBarrier
  \begin{table}[H] 
    \begin{center}
      \begin{tabular}[pos]{|m{5cm} | m{8cm}|} 
        \hline % este comando coloca uma linha na tabela
        \cellcolor[gray]{0.9}\textbf{Name} & \cellcolor[gray]{0.9}\textbf{Roles/Responsibilities} \\ \hline
        Igo Amauri Luz & TBD \\ \hline
        Lauê Rami Costa & TBD \\ \hline
      \end{tabular}
    \end{center}
  \end{table} 
  
  \subsection{Document Outline Description}	
  
  \subsection{Acronyms and Abbreviations}
  \FloatBarrier
  \begin{table}[H]
    \begin{center}
      \begin{tabular}[pos]{|m{2cm} | m{9cm}|} 
				\hline 
				\cellcolor[gray]{0.9}\textbf{Acronym} & \cellcolor[gray]{0.9}\textbf{Description} \\ \hline
				RISC & Reduced Instruction Set Computer \\ \hline
      \end{tabular}
    \end{center}
  \end{table}  

	%%%%%%%%%%%%%%%%%%%%%%%%%%%%%%%%%%%%%%%%%%%%%%%%%%
	%% Document description
	%%%%%%%%%%%%%%%%%%%%%%%%%%%%%%%%%%%%%%%%%%%%%%%%%%
	\section{DUV Overview}
	
	\section{Verification Environment}
	
	\subsection{Design Under Test Interface (DUT_IF)}
	
	\subsection{Monitor}
	
	\subsubsection{Forwarding}
	
	\subsubsection{Branch Taken}
	
	\subsubsection{Load Hazard}
	
	\subsection{Verification Environment Design Specification}
	
	\section{Features List}
	
	\section{Test List}
	
	\section{Assertions}
	
	\section{Coverage and Completion Requirements}

	\section{Resource Requirements}
	
	\section{Schedule}
	
		
	

\end{document}
